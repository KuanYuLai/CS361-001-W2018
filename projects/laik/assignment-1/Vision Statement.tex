\documentclass[12pt]{article}
%\usepackage{times}
\usepackage{cite}
%this is a comment
\title{Pulse}
\author{Korsnesst\\ laik}
\date{January 15, 2018}



\begin{document}
\maketitle



\section*{Problem:}
People everywhere consistently work to improve their health, but training your body and eating right isn’t always easy. People are often inconsistent with their journey towards health. Part of this is due to lack of knowledge. When people first start going to the gym, they find themselves in a new situation and aren’t sure what machines to use or what muscles to work. Another issue is food intake. It’s difficult to track nutrition intake throughout the day. I have had issues with both these things. Friends of mine have also had the same problems when they started going to the gym. There are resources out there but there scattered around and difficult to find.

\section*{Solution:}
Our solution to this problem is to make an all in one phone app. It will provide exercises for each muscle as well as full workout plans. This will take the guesswork out of the equation for people new to working out. It will also have a nutrition tracker. Just by inputting what type of food and the quantity, it will tell you how many remaining calories you should intake during that day. In addition the app will have a running option which will track the users running route, time, distance, and pace. With these features, the user can easily jump into working out without having such a difficult time learning the ins and outs of the gym. The nutrition tracker will also aid in the users goal of either gaining mass or losing mass.


\section*{Limitations/Resources:}
This app will most likely need internet access in order for the workout list to be accessible. The run tracker should still work without internet. In order for this app to work, it will need to use the gps in the phone to track the route of the user. It will also need access to a large amount of workouts. These could be brought in from a different source or added manually. Another resource needed is food information. The app will need to know how many calories are in most types of food. This will probably need to be obtained from an outside source.

\section*{Challenges:}
A challenge for this app will be to implement all three portions of this app. The workout info, the run tracking, and the calorie tracker. Each of these are very different and require outside resources. In order to avoid issues, the team that works on it should work on one section at a time. In addition to this, the group should have good awareness and know when to drop a portion of the app. If one section is looking very difficult and will take a long time, they should make the decision to remove it from the app in favor of the deadline.

\end{document}